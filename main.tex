% Style for a MSc paper at Warsaw School of Economics
% Michał Ramsza
% Friday, December 14, 2012

% --- document class and other global stuff ---------------------------
\documentclass[polish, twoside, 12pt, a4paper]{article}

%% --- packages --------------------------------------------------------
\usepackage{textcomp}
\usepackage{times}
\usepackage{amsmath}
\usepackage{amsfonts}
\usepackage{amssymb}
\usepackage{amsthm}
\usepackage[T1]{fontenc}
\usepackage[utf8]{inputenc}
\usepackage{graphicx}
\usepackage{xcolor}
\usepackage{enumitem}
\usepackage[polish]{babel}
\usepackage[centering, left=3.5cm, right=2.5cm, textheight=24cm]{geometry}

% --- packages for citations ------------------------------------------
\usepackage{natbib}
\AtBeginDocument{\renewcommand{\harvardand}{i}}

% --- package for automatic insertion of R code -----------------------
\usepackage{listings}
\lstset{language=R,%
   numbers=left,%
   tabsize=3,%
   numberstyle=\footnotesize,%
   basicstyle=\ttfamily \footnotesize \color{black},%
   escapeinside={(*@}{@*)}}

% --- support for links -----------------------------------------------
\usepackage{url}
\usepackage{hyperref}
\hypersetup{colorlinks=true,
            linkcolor=black,
            citecolor=darkgray,
            urlcolor=darkgray,
            pagecolor=darkgray}

% --- support for large tables and other stuff ------------------------
\usepackage{longtable}
% \usepackage{subfigure} % this package will not work with subcaption package
\usepackage{float}
\usepackage{caption}
\usepackage{subcaption}
\usepackage{wrapfig}
\usepackage{pdflscape} % relevant for wide tables (rotating pages)

% --- packages for game theory -----------------------------------------
\usepackage{sgame}

% --- support for no widows --------------------------------------------
\usepackage[defaultlines=4,all]{nowidow}

% --- quotation for polish language \enquote{}
\usepackage[autostyle]{csquotes}
\DeclareQuoteAlias{dutch}{polish}

% --- definitions for environments -------------------------------------
\theoremstyle{definition}
    \newtheorem{condition}{Założenie}
    \newtheorem{example}{Przykład}

\theoremstyle{plain}
    \newtheorem{definition}{Definicja}
    \newtheorem{proposition}{Stwierdzenie}
    \newtheorem{theorem}{Twierdzenie}
    \newtheorem{cor}{Wniosek}

\theoremstyle{remark}
    \newtheorem{remark}{Uwaga}

% --- other settings --------------------------------------------------
\linespread{1.5}
\frenchspacing
\sloppy
\allowdisplaybreaks[4]
\raggedbottom
\clubpenalty=10000
\widowpenalty=10000

% --- only if required ------------------------------------------------
\AtBeginDocument{\renewcommand*{\figurename}{Wykres}}
\AtBeginDocument{\renewcommand*{\tablename}{Tabela}}

% --- changing definition of footnote ---------------------------------
\makeatletter
\renewcommand\footnotesize{%
   \@setfontsize\footnotesize\@ixpt{10}%
   \abovedisplayskip 8\p@ \@plus2\p@ \@minus4\p@
   \abovedisplayshortskip \z@ \@plus\p@
   \belowdisplayshortskip 4\p@ \@plus2\p@ \@minus2\p@
   \def\@listi{\leftmargin\leftmargini
               \topsep 4\p@ \@plus2\p@ \@minus2\p@
               \parsep 2\p@ \@plus\p@ \@minus\p@
               \itemsep \parsep}%
   \belowdisplayskip \abovedisplayskip
}
\makeatother


% ---------------------------------------------------------------------
\begin{document}

% --- strona tytulowa -------------------------------------------------
\begin{titlepage}
\centering

\includegraphics[width=0.66\textwidth]{logo.JPG}

\vspace*{0.5cm}
Studium magisterskie\\
\begin{flushleft}
Kierunek: Metody Ilościowe w Ekonomii i Systemy Informacyjne\\
%Specjalność: <specjalność>
% Forma studiów: <forma studiów (stacjonarne, itd.)>
\end{flushleft}

\vspace*{.5cm}
\rule{0cm}{1cm}\hfill
\begin{minipage}{9cm}
Imie i nazwisko autora: Oskar Furmańczuk\\
Nr albumu: 81794
\end{minipage}

\vspace*{1cm}
\begin{minipage}{12cm}
\centering
\Large
\textbf{Determinanty przebiegu choroby COVID-19.}
\end{minipage}

\vspace*{2cm}
\rule{0cm}{1cm}\hfill
\begin{minipage}{9cm}
Praca licencjacka napisana\\
w Katedrze Matematyki i Ekonomii Matematycznej\\
pod kierunkiem naukowym\\
dr hab. Michała Ramszy
\end{minipage}

\vfill
Warszawa 2021
\end{titlepage}

\rule{1ex}{0ex}\clearpage


% --- table of contents -----------------------------------------------
\cleardoublepage
\tableofcontents

% --- chapter ---------------------------------------------------------
\cleardoublepage
\section{Wprowdzenie}

- przegląd litertury dotyczącej koronawirusa - artykuły poświęcone badaniom wpływu czynników biologicznych na przebieg choroby \\
- związek pracy z ekonomią \\

% --- chapter ---------------------------------------------------------
\clearpage
\section{Opis zbioru danych}

- liczebność oraz źródło zbioru danych \\
- opisy zmiennych \\
- rozkład zmiennych \\
- charakterystyki zmiennych (np. średnie, odchylenia std ) \\
- charakterystyka brakujących danych \\
- problemy wynikające z brakujących danych


% --- chapter ---------------------------------------------------------
\clearpage
\section{Metody}

- opis kolejności obliczeń/transformacji/budowy modelu \\
- opis użytego modelu (docelowo: xgboost) oraz jego zasadność względem niezbalansowanej próby\\
- opis metod walidacji oraz jakości modelu (F1-score, accuracy) \\
- opis sposobu interpretacji wpływu zmiennych objaśniających na zmienną objaśnianą (docelowo: SHAP)

% --- chapter ---------------------------------------------------------
\clearpage
\section{Wyniki i dyskusja}

- przedstawienie wzkaźników jakości modelu \\
- przstawienie wyników SHAP oraz ich interpretacja \\
- porównanie wniosków uzyskanych z opracowanego modelu z wnioskami z przywołanej literatury naukowej \\ 

% --- chapter ---------------------------------------------------------
\clearpage
\section{Zakończenie}


- podsumowanie pracy \\


% --- chapter -------
--------------------------------------------------
\clearpage
\section{Literatura}



% --- appendices ------------------------------------------------------
\appendix

% ---------------------------------------------------------------------
\clearpage
\section{Dodatek: Ważne rzeczy do dodania}




% --- bibliography ----------------------------------------------------
\clearpage
\bibliographystyle{agsm}
\bibliography{refs}

% --- abstract --------------------------------------------------------
\clearpage
\addcontentsline{toc}{section}{Lista tablic}
\listoftables

% --- abstract --------------------------------------------------------
\clearpage
\addcontentsline{toc}{section}{Lista rysunków}
\listoffigures



% --- abstract --------------------------------------------------------
\clearpage
\addcontentsline{toc}{section}{Streszczenie}
\section*{Streszczenie}

Tutaj zamieszczają Państwo streszczenie pracy. Streszczenie powinno być długości około pół strony.


\end{document}

%%% Local Variables:
%%% mode: latex
%%% TeX-master: t
%%% End:
